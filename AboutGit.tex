\documentclass[12pt]{article} 
\usepackage[xetex, a4paper, left=2cm, right=2cm, top=2cm,bottom=2cm]{geometry}
\usepackage[cm-default]{fontspec}
\usepackage{xunicode}

%\tolerance=1000
%\emergencystretch=0.74cm 

\usepackage{polyglossia}
\setdefaultlanguage[spelling=modern]{russian}
\setotherlanguage{english} 
\defaultfontfeatures{Scale=MatchLowercase,Ligatures=TeX}  %% устанавливает поведение шрифтов по умолчанию  
\newfontfamily\cyrillicfont{Linux Libertine} 
\setromanfont[Mapping=tex-text]{Linux Libertine}
\setsansfont[Mapping=tex-text]{Linux Biolinum}
\setmonofont{Ubuntu Mono}
%\newfontfamily\cyrillicfont{Liberation Mono} 

%\usepackage{makecell}

%\usepackage{titlesec}
%\newcommand{\sectionbreak}{\clearpage}

%\renewcommand{\thesection}{\Alph{section}}
%\newcount\wd    \wd=\textwidth \multiply\wd by 8 \divide\wd by 17

\usepackage{minted}
\usemintedstyle{friendly}
\newminted{bash}{frame=lines}
\newminted{c}{frame=leftline}
\renewcommand\listingscaption{Код}


\usepackage[unicode, pdfborder={0 0 0 0}]{hyperref}

\author{Alaksiej Stankievič}
\title{Базовый модуль}

\begin{document}
\hypersetup{
pdftitle = {Основные команды git},
pdfauthor = {Alaksiej Stankievič},
pdfsubject = {базовый модуль}}% End of hypersetup

\section{Команды bash}
Для смены папки используются команда \verb|cd| (код \ref{lst:cd}). Имя папки может либо быть полным (тогда оно начинается от корня \verb|/|), либо относительным ищется в данной папке. Есть два особых имени папок: \verb|.| текущая папка и \verb|..| папка на уровень выше, чем можно пользоваться (код \ref{lst:cdtwodot}). 
\begin{listing}[H]
\begin{center}
\begin{bashcode}
cd <имя папки>
\end{bashcode}
\end{center}
\caption{Смена директории}
\label{lst:cd}
\end{listing}

\begin{listing}[H]
\begin{center}
\begin{bashcode}
cd ..
\end{bashcode}
\end{center}
\caption{Подняться на директорию вверх}
\label{lst:cdtwodot}
\end{listing}

Для просмотра содержимого используется команда \verb|ls|. Без опций она выводит просто содержимое текущей папки. У многих команд есть опция \verb|--help|, которая выдаёт справку по использованию данной команды. Часто распространённые опции можно сократить до одной буквы, например, \verb|-h| (внимание! в случае \verb|ls| это не справка, а человекочитаемый формат размера файла\footnote{полезна в сочетании с опцией -l}). Так у команды \verb|ls| есть опция \verb|-a| --- показ скрытых файлов, опция \verb|-l| --- выведение в виде списка с более подробной информацией и уже упомянутая опция \verb|-h|, их все можно сочетать так как показано в коде \ref{lst:lsalh}.

\begin{listing}[H]
\begin{center}
\begin{bashcode}
ls -alh
\end{bashcode}
\end{center}
\caption{Показать содержимое текущей папки включая скрытые файлы в виде списка, размер файла выводить в человекочитаемом формате}
\label{lst:lsalh}
\end{listing}

Для создания папки используется команда \verb|mkdir| (код \ref{lst:mkdir}).
\begin{listing}[H]
\begin{center}
\begin{bashcode}
mkdir <имя папки>
\end{bashcode}
\end{center}
\caption{Создание папки}
\label{lst:mkdir}
\end{listing}

Для копирования и перемещения(переименования) используются команды \verb|cp| и \verb|mv| соответственно с похожим синтаксисом (код \ref{lst:cpmv}).
\begin{listing}[H]
\begin{center}
\begin{bashcode}
cp <исходное имя> <итоговое имя>
mv <исходное имя> <итоговое имя>
\end{bashcode}
\end{center}
\caption{Копирование и перемещение}
\label{lst:cpmv}
\end{listing}

Для удаления используется команда \verb|rm|. Если нужно удалить папку нужно употребить опции \verb|-r| и \verb|-f| (код \ref{lst:rmrf}).
\begin{listing}[H]
\begin{center}
\begin{bashcode}
rm -rf <имя папки>
\end{bashcode}
\end{center}
\caption{Удаление папки}
\label{lst:rmrf}
\end{listing}

\section{Команды git}
\subsection{Редкие команды}

Для гита важно понятие репозитория (\textit{repository}). Это папка в которой находится папки и файлы и её содержимое находится под системой контроля версий (СКВ). Так как гит распределённая СКВ, то каждый репозиторий полноправный (на локальном компьютере или на гитхабе, единственное отличие, что у вас нет возможности попасть в репозиторий на гитхабе напрямую, и вы взаимодействуете с ним как с удалённым (в смысле далеко) репозиторием). Пустой репозиторий можно создать только один раз, это делает команда \verb|init| в папке будущего репозитория (код \ref{lst:gitinit}).
\begin{listing}[H]
\begin{center}
\begin{bashcode}
git init
\end{bashcode}
\end{center}
\caption{Создание пустого репозитория}
\label{lst:gitinit}
\end{listing}

К созданному репозиторию нужно присоединить удалённый репозиторий созданный на гитхабе. Это делается следующим кодом:
\begin{listing}[H]
\begin{center}
\begin{bashcode}
git remote add github <здесь url с гитхаба>
\end{bashcode}
\end{center}
\caption{Присоединение удалённого репозитория}
\label{lst:gitremote}
\end{listing}

Просмотреть существующие удалённые репозитории можно командой: 
\begin{listing}[H]
\begin{center}
\begin{bashcode}
git remote --verbose
\end{bashcode}
\end{center}
\caption{Просмотр удалённых репозиториев}
\label{lst:gitremotelist}
\end{listing}

На остальных компьютерах репозиторий не создаётся, а клонируется не пустой репозиторий с гитхаба (внимание! это единственная команда, которая выполняется не в папке репозитория, а в папке НАД будущим репозиторием):
\begin{listing}[H]
\begin{center}
\begin{bashcode}
git clone -o github <здесь url с гитхаба>
\end{bashcode}
\end{center}
\caption{Клонирование существующего репозитория}
\label{lst:gitclone}
\end{listing}

В любом случае у вас должен быть настроен пользователь, в компьютерном классе это нужно делать для каждого репозитория (и поэтому каждый раз при создании репозитория), чтобы не мешать другим студентам, а дома это достаточно сделать один раз с опцией \verb|--global| (код \ref{lst:gitconfig}). Просмотреть настройки можно теми же командами не указывая значения имени или почты.
\begin{listing}[H]
\begin{center}
\begin{bashcode}
git config user.name <имя пользователя>
git config user.email <e-mail пользователя>
\end{bashcode}
\end{center}
\caption{Настройка пользователя}
\label{lst:gitconfig}
\end{listing}

\subsection{Частые команды}
\subsubsection{Команды <<коммита>>}

Файлы в СКВ могут находится в 3 состояниях: сыром, подготовленном и зафиксированном (<<закоммиченном>>). Файл из сырого в подготовленное состояние переводится командой \verb|add| (код \ref{lst:gitadd}), новые файлы можно подготовить только через эту команду, изменённые файлы уже находящиеся под СКВ можно автоматически добавить при коммите  использую опцию \verb|-a| (код \ref{lst:gitcommitam}). Все подготовленные на данный момент файлы можно зафиксировать (<<закоммитить>>) c помощью команды \verb|commit|, так как фиксируется подготовленные на данный момент файлы, то не нужно указывать имена файлов (код \ref{lst:gitcommit}). Каждый коммит обязан иметь имя, его можно указать с помощью опции \verb|-m|, если вы не укажете имя коммита, гит откроет текстовый редактор для набора, по умолчанию гит настроен на работу с редактором vim, который не для новичков\footnote{Если вы всё же в него попали то выйти можно так: наберите ':', а затем 'q' и '!', нажмите ввод.}\footnote{Для тех, кто хочет научится работать в vim'e советую нагуглить как установить в гитбаш vimtutor. В виме есть понятия режима, в котором клавиши ведут себя по разному, изначально он находится в режиме команд. Он переходит в режим вставки по клавише 'i' (английской!), а выходи по клавише 'esc'. В режиме команд можно удалять символ под курсором клавишей 'x'. В режиме команд после двоеточия предлагаются специальные команды, как уже писал 'q!' выход без сохранения, выход с сохранением 'wq'. Это минимальный набор по выживанию, а команд там очень, очень много.}. Если редактировать зафиксированный файл (в кодеблоке или просто), то он перейдёт в сырое состояние.

\begin{listing}[H]
\begin{center}
\begin{bashcode}
git add <имя файла или папки>
\end{bashcode}
\end{center}
\caption{Подготовка файла (папки)}
\label{lst:gitadd}
\end{listing}

\begin{listing}[H]
\begin{center}
\begin{bashcode}
git commit -m '<название коммита>'
\end{bashcode}
\end{center}
\caption{Коммит}
\label{lst:gitcommit}
\end{listing}

\begin{listing}[H]
\begin{center}
\begin{bashcode}
git commit -am '<название коммита>'
\end{bashcode}
\end{center}
\caption{Добавление изменений и коммит}
\label{lst:gitcommitam}
\end{listing}

\subsubsection{Команды работы с удалённым репозиторием}
Вам необходимо отправлять и забирать коммиты на гитхаб и с гитхаба это осуществляется командами \verb|push| (отправить) и \verb|pull| (забрать). Они обладают похожим синтаксисом (код \ref{lst:gitpushpull}). Мы пока работаем на основной ветке, поэтому если вы делали всё как по инструкции, то вот так выглядит отправка (код \ref{lst:gitpush}), а так стягивание (код \ref{lst:gitpull}). Всегда выполняйте перед работой стягивание (например, вы отправили что-то в шагае, а дома это не забрали и стали работать, что-то накомитили, гитхаб отклонит ваш push в таком случае (но это разрешимая ситуация)). Признак того, что всё удачно отправилось, сообщение в конце которого есть фраза со стрелочкой.

\begin{listing}[H]
\begin{center}
\begin{bashcode}
git <push|pull> <имя удалённого репозитария> <имя ветки>
\end{bashcode}
\end{center}
\caption{Синтаксис отправки и стягивания}
\label{lst:gitpushpull}
\end{listing}

\begin{listing}[H]
\begin{center}
\begin{bashcode}
git push github master
\end{bashcode}
\end{center}
\caption{Синтаксис отправки}
\label{lst:gitpush}
\end{listing}

\begin{listing}[H]
\begin{center}
\begin{bashcode}
git pull github master
\end{bashcode}
\end{center}
\caption{Синтаксис стягивания}
\label{lst:gitpull}
\end{listing}

\subsubsection{Команды просмотра состояния}
Система контроля версий --- это дорога с односторонним движением. Её предназначение фиксировать истинное состояние в истории, и если вы что-то зафиксировали с ошибками, значит так оно и было. И хотя существуют пути по исправлению истории, они специально сделаны трудоёмкими, чтобы уменьшить количество фальсификаторов истории (в корпоративной разработке можно настроить СКВ так, что это вообще невозможно). Поэтому семь раз подумай, один раз закоммить. Помните, что исправлять ошибки, которые живут пока только в локальном репозитории гораздо проще, чем если вы их уже закинули на гитхаб. Поэтому ещё три раза всё проверь, а только потом отправь.

\begin{center}
\large\framebox{Система контроля версий --- это дорога с односторонним движением.}
\end{center}

\begin{center}
\large\framebox{Семь раз подумай, один раз закоммить.}
\end{center}

\begin{center}
\large\framebox{Ещё три раза всё проверь, а только потом отправь.}
\end{center}

Чтобы всё это делать нужны команды просмотра состояний. Команда \verb|status| показывает, что у вас творится в репозитории есть ли сырые или подготовленные файлы, не ушли ли вы вперёд по сравнению з гитхабом, и многое другое (она также даёт советы, что можно сделать). Команда \verb|log| показывает список коммитов (его можно листать стрелочками). Команда \verb|show| показывает последний коммит (либо любой ели вы укажете хэш коммита который узнали из лога), здесь тоже работают стрелочки, и здесь как и в логе можно выйти нажав q.

\begin{center}
Удачных всем коммитов!
\end{center}

\subsection{Файл gitignore}

Файл гитигнор предназначен для игнорирования файлов определённого вида от отслеживания СКВ. Он подчиняется принципам глобинг шаблонов юникса, за некоторыми нюансами. Символ '*' заменяет любой символ или набор символов, в том числе отсутствующий. То что начинается с косой черты '/' интерпретируется как непосредственная папка в репозитории, а то что заканчивается на косую черту интерпретируется  как произвольная папка. Символы взятые в квадратные скобки интерпретируются как 1 любой из этих символов. Файлы попадающие под шаблон из gitignore не отслеживаются СКВ, он должен начинаться с точки и не содержать расширения (\verb|.gitignore|).

\begin{listing}[H]
\begin{center}
\begin{bashcode}
*.zuzu
\end{bashcode}
\end{center}
\caption{Игнорирование файла с расширением zuzu}
\label{lst:ignoreext}
\end{listing}

\begin{listing}[H]
\begin{center}
\begin{bashcode}
*zuzu*
\end{bashcode}
\end{center}
\caption{Игнорирование файла содержащего zuzu в имени}
\label{lst:ignorepatern}
\end{listing}

\begin{listing}[H]
\begin{center}
\begin{bashcode}
/zuzu
\end{bashcode}
\end{center}
\caption{Игнорирование папки zuzu в корневой директории}
\label{lst:ignorerootfolder}
\end{listing}

\begin{listing}[H]
\begin{center}
\begin{bashcode}
zuzu/
\end{bashcode}
\end{center}
\caption{Игнорирование любой папки zuzu}
\label{lst:ignorefolder}
\end{listing}

\begin{listing}[H]
\begin{center}
\begin{bashcode}
z[aou]zu
\end{bashcode}
\end{center}
\caption{Игнорирование zazu, zozu, и zuzu}
\label{lst:ignorebracket}
\end{listing}

\end{document}
