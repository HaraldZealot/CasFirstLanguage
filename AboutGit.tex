\documentclass[12pt]{article} 
\usepackage[xetex, a4paper, left=2cm, right=2cm, top=2cm,bottom=2cm]{geometry}
\usepackage[cm-default]{fontspec}
\usepackage{xunicode}

%\tolerance=1000
%\emergencystretch=0.74cm 

\usepackage{polyglossia}
\setdefaultlanguage[spelling=modern]{russian}
\setotherlanguage{english} 
\defaultfontfeatures{Scale=MatchLowercase,Ligatures=TeX}  %% устанавливает поведение шрифтов по умолчанию  
\newfontfamily\cyrillicfont{Linux Libertine} 
\setromanfont[Mapping=tex-text]{Linux Libertine}
\setsansfont[Mapping=tex-text]{Linux Biolinum}
\setmonofont{Ubuntu Mono}
%\newfontfamily\cyrillicfont{Liberation Mono} 

%\usepackage{makecell}

%\usepackage{titlesec}
%\newcommand{\sectionbreak}{\clearpage}

%\renewcommand{\thesection}{\Alph{section}}
%\newcount\wd    \wd=\textwidth \multiply\wd by 8 \divide\wd by 17

\usepackage{minted}
\usemintedstyle{friendly}
\newminted{bash}{frame=lines}
\newminted{c}{frame=leftline}
\renewcommand\listingscaption{Код}


\usepackage[unicode, pdfborder={0 0 0 0}]{hyperref}

\author{Alaksiej Stankievič}
\title{Базовый модуль}

\begin{document}
\hypersetup{
pdftitle = {Основные команды git},
pdfauthor = {Alaksiej Stankievič},
pdfsubject = {базовый модуль}}% End of hypersetup

\section{Команды bash}
Для смены папки используются команда \verb|cd| (код \ref{lst:cd}). Имя папки может либо быть полным (тогда оно начинается от корня \verb|/|), либо относительным ищется в данной папке. Есть два особых имени папок: \verb|.| текущая папка и \verb|..| папка на уровень выше, чем можно пользоваться (код \ref{lst:cdtwodot}). 
\begin{listing}[H]
\begin{center}
\begin{bashcode}
cd <имя папки>
\end{bashcode}
\end{center}
\caption{Смена директории}
\label{lst:cd}
\end{listing}

\begin{listing}[H]
\begin{center}
\begin{bashcode}
cd ..
\end{bashcode}
\end{center}
\caption{Подняться на директорию вверх}
\label{lst:cdtwodot}
\end{listing}

Для просмотра содержимого используется команда \verb|ls|. Без опций она выводит просто содержимое текущей папки. У многих команд есть опция \verb|--help|, которая выдаёт справку по использованию данной команды. Часто распространённые опции можно сократить до одной буквы, например, \verb|-h| (внимание! в случае \verb|ls| это не справка, а человекочитаемый формат размера файла\footnote{полезна в сочетании с опцией -l}). Так у команды \verb|ls| есть опция \verb|-a| --- показ скрытых файлов, опция \verb|-l| --- выведение в виде списка с более подробной информацией и уже упомянутая опция \verb|-h|, их все можно сочетать так как показано в коде \ref{lst:lsalh}.

\begin{listing}[H]
\begin{center}
\begin{bashcode}
ls -alh
\end{bashcode}
\end{center}
\caption{Показать содержимое текущей папки включая скрытые файлы в виде списка, размер файла выводить в человекочитаемом формате}
\label{lst:lsalh}
\end{listing}

Для создания папки используется команда \verb|mkdir| (код \ref{lst:mkdir}).
\begin{listing}[H]
\begin{center}
\begin{bashcode}
mkdir <имя папки>
\end{bashcode}
\end{center}
\caption{Создание папки}
\label{lst:mkdir}
\end{listing}

Для копирования и перемещения(переименования) используются команды \verb|cp| и \verb|mv| соответственно с похожим синтаксисом (код \ref{lst:cpmv}).
\begin{listing}[H]
\begin{center}
\begin{bashcode}
cp <исходное имя> <итоговое имя>
mv <исходное имя> <итоговое имя>
\end{bashcode}
\end{center}
\caption{Копирование и перемещение}
\label{lst:cpmv}
\end{listing}

Для удаления используется команда \verb|rm|. Если нужно удалить папку нужно употребить опции \verb|-r| и \verb|-f| (код \ref{lst:rmrf}).
\begin{listing}[H]
\begin{center}
\begin{bashcode}
rm -rf <имя папки>
\end{bashcode}
\end{center}
\caption{Удаление папки}
\label{lst:rmrf}
\end{listing}

\section{Команды git}


\begin{listing}[ht]
\begin{center}
\begin{bashcode}
git init
\end{bashcode}
\end{center}
\caption{Создание пустого репозитория}
\label{lst:gitinit}
\end{listing}



\end{document}
