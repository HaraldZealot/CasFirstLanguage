\documentclass[12pt]{article} 
\usepackage[xetex, a4paper, left=2cm, right=2cm, top=2cm,bottom=2cm]{geometry}
\usepackage[cm-default]{fontspec}
\usepackage{xunicode}

%\tolerance=1000
%\emergencystretch=0.74cm 

\usepackage{polyglossia}
\setdefaultlanguage[spelling=modern]{russian}
\setotherlanguage{english} 
\defaultfontfeatures{Scale=MatchLowercase,Ligatures=TeX}  %% устанавливает поведение шрифтов по умолчанию  
\newfontfamily\cyrillicfont{Linux Libertine} 
\setromanfont[Mapping=tex-text]{Linux Libertine}
\setsansfont[Mapping=tex-text]{Linux Biolinum}
\setmonofont{Ubuntu Mono}
%\newfontfamily\cyrillicfont{Liberation Mono} 

%\usepackage{makecell}

%\usepackage{titlesec}
%\newcommand{\sectionbreak}{\clearpage}

%\renewcommand{\thesection}{\Alph{section}}
%\newcount\wd    \wd=\textwidth \multiply\wd by 8 \divide\wd by 17

\usepackage{minted}
\usemintedstyle{friendly}
\renewcommand\listingscaption{Код}


\usepackage[unicode, pdfborder={0 0 0 0}]{hyperref}

\author{Alaksiej Stankievič}
\title{Базовый модуль}

\begin{document}
\hypersetup{
pdftitle = {Язык программирования С. Базовый модуль},
pdfauthor = {Alaksiej Stankievič},
pdfsubject = {базовый модуль}}% End of hypersetup

\section{Команды баша}
Для смены папки используются команда \verb|cd| (код \ref{lst:cd}). Имя папки может либо быть полным (тогда оно начинается от корня \verb|/|), либо относительным ищется в данной папке. Есть два особых имени папок: \verb|.| текущая папка и \verb|..| папка на уровень выше, чем можно пользоваться (код \ref{lst:cdtwodot}). 
\begin{listing}[ht]
\begin{center}
\begin{minted}{bash}
cd <имя папки>
\end{minted}
\end{center}
\caption{Смена директории}
\label{lst:cd}
\end{listing}

\begin{listing}[ht]
\begin{center}
\begin{minted}{bash}
cd ..
\end{minted}
\end{center}
\caption{Подняться на директорию вверх}
\label{lst:cdtwodot}
\end{listing}

\section{Команды гита}


\begin{listing}[ht]
\begin{center}
\begin{minted}{bash}
git init
\end{minted}
\end{center}
\caption{Создание пустого репозитория}
\label{lst:gitinit}
\end{listing}



\end{document}
